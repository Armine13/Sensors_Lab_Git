\documentclass[english]{article}
\usepackage[T1]{fontenc}
\usepackage[latin9]{inputenc}
\usepackage{babel}
\usepackage{graphicx}
\usepackage{subfigure}
\usepackage{float}
\setlength{\parindent}{0pt}
\usepackage{amsmath}


\begin{document}

\title{Lab 1: Infrared Imaging\\ -------------------------------- \\ \Large Sensors and Digitization}
\author{ \ Armine Vardazaryan, Songyou Peng \\ arminevardazaryan@gmail.com, psy920710@gmail.com}
\date{26th November 2015}

\maketitle

\section{Introduction}

\section{Simplified polarization imaging}

\subsection{Wolff's method}


\section{Contrast polarization measurement}
\subsection{Polarizers and a rotator}
\section{Conclusion}

\section{References}
{[}1{]} Wolff, Lawrence B. "Polarization vision: a new sensory approach to image understanding." Image and Vision computing 15.2 (1997): 81-93.\\
{[}2{]} Goldstein, Dennis H. Polarized light. CRC Press, 2010.\\
{[}3{]} -https://en.wikipedia.org/wiki/Polarizer\\


\end{document}