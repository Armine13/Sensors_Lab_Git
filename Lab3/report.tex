\documentclass[english]{article}
\usepackage[T1]{fontenc}
\usepackage[latin9]{inputenc}
\usepackage{babel}
\usepackage{graphicx}
\usepackage{subfigure}
\usepackage{float}
\setlength{\parindent}{0pt}
\usepackage{amsmath}


\begin{document}

\title{Lab 3: Lens \& Lighting\\ -------------------------------- \\ \Large Sensors and Digitization}
\author{ \ Armine Vardazaryan, Songyou Peng \\ arminevardazaryan@gmail.com, psy920710@gmail.com}
\date{5rd December 2015}

\maketitle

\section{Introduction}
This lab offers us a chance to have a insight of lens and light.
In the first part, we will choose different lens and extension rings, and then talk about the influence on the acquisition.
The second part is mainly about the observation of some objects under different light source, and do defect detection. 

\section{Lens}
At the very beginning of this lab, we want to compute the "Spatial Resolution". In the Figure \ref{fig:one}, we try to demonstrate the length of this fork by putting a ruler next to it.
We can see the length of the pork is about 214 $mm$ and the number of pixels between this length is $(1310 - 40) = 1270$, so we can easily get the spatial resolution as below:
$$
resolution = \frac{number of pixels}{length} = \frac{1270}{214} = 5.935\ pixel/mm
$$
\begin{figure}[H]
	\centering
	\includegraphics[width=1\linewidth]{Pictures/spatial_resolution.png}
	\caption{Picture for computing spatial resolution}
	\label{fig:one}
\end{figure} 

Now we will compute the focal length of the lens to test whether we use a correct lens or not.
In order to get Figure \ref{fig:one}, our intial working distance is 500$mm$.
And we know our CCD height and width are 6.6$mm$ and 8.8$mm$ respectively.
So after applying the formula:
\begin{align*} 
	Focal\ length\ of\ the\ height &= \frac{working\ distance * CCD\ height}{Object\ height + CCD\ height}\\
	&= \frac{500 \times 6.6}{214 + 6.6} \\
	&= 15.00mm
\end{align*}
It turns out we use correct lens because the 16$mm$ length is the closest lens that we have to 15mm.
\section{Extension Rings}
In this part, we try various lens and rings in order to observe the influence of extension ring.
We know that extension ring will get the lens further away from the focal plane, which leads to the decrease of working distance.
So with an extension ring, you can take a closer picture of your object and include more information and more part of the object.\\
\\
First, we try 50$mm$ lens with 5$mm$ ring, image shown in Figure \ref{fig:twoa}.
At this time, working distance that we measure is 300$mm$.
When we remove the ring, the image is Figure \ref{fig:twob} and working distance now is 515$mm$. Obviously, more information and bigger fork we get in the Figure \ref{fig:twoa}.
\begin{figure}[H]
	\centering
	\subfigure[Object with 5$mm$ ring]{\label{fig:twoa}
	\includegraphics[width=0.45\linewidth]{Pictures/ring_50.png}
	}
	\subfigure[Object without ring]{\label{fig:twob}
	\includegraphics[width=0.45\linewidth]{Pictures/without_ring_50.png}
	}
	\caption{Object with 50$mm$ lens}
	\label{fig:two}
\end{figure}

In another example we use 35$mm$ lens and compare coin images by using different extension rings, which is shown in Figure \ref{fig:three}.
It is easy to be noticed that the image with 20$mm$ ring contains more information than the one with 5$mm$ ring.\\
\\
So, if you want to put the object full of your image, choosing a proper extension ring would be helpful.
\begin{figure}[H]
	\centering
	\subfigure[Object with 5$mm$ ring]{\label{fig:threea}
	\includegraphics[width=0.45\linewidth]{Pictures/bonus/coin35+5.png}
	}
	\subfigure[Object with 20$mm$ ring]{\label{fig:threeb}
	\includegraphics[width=0.45\linewidth]{Pictures/bonus/35+20_coin.png}
	}
	\caption{Object with 35$mm$ lens}
	\label{fig:three}
\end{figure}

\section{Lighting}


\section{Conclusion}


\section{References}
{[}1{]} - https://photographylife.com/what-is-an-extension-tube

\end{document}